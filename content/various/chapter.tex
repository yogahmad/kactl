\chapter{Various}

\section{Known Problems}
\kactlimport{StableMarriage.h}
\kactlimport{FlowShopScheduling.h}
% \kactlimport{2sat.h}
\kactlimport{KonigTheorem.h}
\kactlimport{MoserCircle.h}
\kactlimport{ChickenMcNugget.h}
\kactlimport{EulerFaceFormula.h}
\kactlimport{CayleyFormula.h}
\kactlimport{PickTheorem.h}
\kactlimport{JosephusProblem.h}
\kactlimport{ErdosGallai.h}

\section{Polynomial}
order unit(x,p) is smallest z such that
\[ x^z = 1 mod p \]
if order unit is p-1, then x is primitive root of p. \newline
FFT mod besar: pecah jadi 2.
\[ A = A1 + 2^{16}*A2 \]
\[ B = B1 + 2^{16}*B2 \]
\[ AB = A1B1 + 2^{16}(A1B2 + A2B1) + 2^{32}A2B2 \]
\[x^{-n} mod Q(X)\] merupakan banyaknya cara tepat n. dimana Q(X) adalah 1/Q(X) pada GF diatas.
\[x^{-1} mod Q(X)\] bernilai $+cx^{i-1}$ untuk setiap $-cx^{i}$ di Q(X). untuk $x^{-n}$, tinggal DnC.

\section{Minimum-Cut Problems}
\subsection{Open-pit Mining}
Given DAG with value on each node (value can be negative). We can choose a node X to be removed from graph and then 
we get the value of that node to the score. Before removing node X, X SHOULD HAVE ZERO in-degree. After removing node X, all edges that outcoming from X become removed. We can do that until the graph has no node, or we can stop at any time.
What is the maximum score? Solution: Sum of positive value on each node plus minimum cut. For every edge (a,b), Connect b to a with capacity INF. Connect every source to positive node with capacity abs(value). Connect every negative node to sink with capacity abs(value).
\subsection{Landscaping}
Given N x N grid, each cell is either low or high land. There are N trucks going straight from top to bottom and N truck going straight from left to right.
Also a cost C is added if a truck hit a land from low to high or high to low. 
We can also toggle the high of the cell and a cost D will be added. 
What is the minimum of total cost? final position of f2N trucks should be at bottom and right. Solution: Connect every cell to its neighbor with capacity C. Connect every low cell to source with capacity D. Connect every high cell to sink with capacity D.

\section{Desperate Optimization}
\kactlimport{FastRead.h}
\kactlimport{FastMod.h}
\kactlimport{ClockTime.h}

% \section{Intervals}
% \kactlimport{IntervalContainer.h}
% \kactlimport{IntervalCover.h}

\section{Misc. algorithms}
% \kactlimport{TernarySearch.h}
\kactlimport{Karatsuba.h}

\section{Dynamic programming}
\kactlimport{DivideAndConquerDP.h}
\kactlimport{KnuthDP.h}

\subsection{Slope Trick Narrow Rectangle}
Define \texttt{dp[i][x]}: the minimum cost to move the first i rectangles such that the last (the i-th) rectangle’s
leftmost coordinate is x. This will lead to a solution for partial score.
Now, see \texttt{dp[i]} as a function: it returns \texttt{dp[i][x]} for given x. It turns out that this function is a polyline
consisting of 2i + 3 sections, and the slopes of the sections are $ -i - 1, -i, . . . , i, i + 1 $ from left to right.
Thus, this polyline can be represented using 2i + 3 integers $l0, l1, ..., li, r0, r1, ..., ri, c$.
\begin{itemize}
	\item In the interval $(-\infty, li]$, the slope of the polyline is $-i - 1$.
	\item In the interval $[li, li-1]$, the slope of the polyline is $-i$.
	\item · · ·
	\item In the interval $[l1, l0]$, the slope of the polyline is $-1$.
	\item In the interval $[l0, r0]$, the polyline is constant, and the value is $c$.
	\item In the interval $[r0, r1]$, the slope of the polyline is $1$.
	\item · · ·
	\item In the interval $[ri-1, ri]$, the slope of the polyline is $i$.
	\item In the interval $[ri,\infty)$, the slope of the polyline is $i + 1$.
\end{itemize}

Now we compute the polylines in the order $dp[0], dp[1], . . . , dp[N]$. We should keep two sets (or priority
queues) representing ${l0,...,li}$ and ${r0,...,ri}$ meanwhile, and the solution works in $O(NlogN)$. 

\kactlimport{NarrowRectangle.h}

\subsection{Slope Trick Explanation}
Define $f_{i}(x)$ as the minimum number of moves to change the first $i$ elements into a non-decreasing sequence such that $ai \leq x$.
It is easy to see that by definition we have the recurrences:
$f_{i}(X) = min_{Y \leq X}(|a_{i}-Y|)$ when $i=1$
and
$f_{i}(X) = min_{Y \leq X}(f_{i-1}+|a_{i}-Y|)$

Now, note that $f_{i}(X)$ is non-increasing, since it is at most the minimum among all the values of f for smaller X by definition. We store a set of integers that denotes where the function fi change slopes. More formally, we consider the function $g_{i}(X) = f_{i}(X+1)-f_{i}(X)$. The last element of the set will be the smallest j such that $g_{i}(j)=0$, the second last element will be the smallest j such that $g_{i}(j)=-1$, and so on. 
(note that the set of slope changing points is bounded).

Let $Opt(i)$ denote a position where $f_{i}(X)$ achieves its minimum. (i.e. $g_{i}(Opt(i)) = 0$) The desired answer will be $f_{n}(Opt(n))$. We'll see how to update these values quickly.

There are two cases to consider: \\
Case 1: $Opt(i-1) \leq a_{i}$ \\
Here, the slope at every point before ai decreases by 1. Thus, we push ai into the slope array as this indicates that we decreases the slope at all the slope changing points by 1, 
and the slope changing point for $slope=0$ is ai, i.e. $Opt(i) = a_{i}$. Thus, this case is settled. \\

Case 2: $Opt(i-1) > a_{i}$ \\
Now, we insert $a_{i}$ into the set, since it decreases the slope at all the slope changing points before $a_{i}$ by 1. 
Furthermore, we insert $a_{i}$ again because it increases the slope at the slope changing points between $a_{i}$ and $Opt(i-1)$ by 1.
Now, we can just take $Opt(i) = Opt(i-1)$ since the slope at $Opt(i-1)$ is still 0. Finally, we remove $Opt(i-1)$
from the set because it's no longer the first point where the slope changes to 0. 
(it's the previous point where the slope changes to $-1$ and the slope now becomes 0 because of the addition of $a_{i}$)
Thus, the set of slope changing points is maintained. We have $f_{i}(Opt(i)) = f_{i-1}(Opt(i-1))+|Opt(i-1)-a_{i}|$.
Thus, we can just use a priority queue to store the slope changing points and it is easy to see that the priority queue can handle all these operations efficiently (in $O(log N)$ time).


\kactlimport{SimpleSlopeTrick.h}

% \section{Debugging tricks}
% \begin{itemize}
% 	\item \texttt{signal(SIGSEGV, [](int) \{ \_Exit(0); \});} converts segfaults into Wrong Answers.
% 	      Similarly one can catch SIGABRT (assertion failures) and SIGFPE (zero divisions).
% 	      \texttt{\_GLIBCXX\_DEBUG} violations generate SIGABRT (or SIGSEGV on gcc 5.4.0 apparently).
% 	\item \texttt{feenableexcept(29);} kills the program on NaNs (\texttt 1), 0-divs (\texttt 4), infinities (\texttt 8) and denormals (\texttt{16}).
% \end{itemize}

\section{Optimization tricks}
\subsection{Bit hacks}
\begin{itemize}
	\item \texttt{x \& -x} is the least bit in \texttt{x}.
	\item \texttt{for (int x = m; x; ) \{ --x \&= m; ... \}} loops over all subset masks of \texttt{m} (except \texttt{m} itself).
	\item \texttt{c = x\&-x, r = x+c; (((r\^{}x) >> 2)/c) | r} is the next number after \texttt{x} with the same number of bits set.
	\item \texttt{ rep(b,0,K) rep(i,0,(1 << K)) if (i \& 1 << b) D[i] += D[i\^{}(1 << b)]; } computes all sums of subsets.
\end{itemize}

BIT Range Sum Update \\ 
Update Query \\
Update(BITree1, l, val) \\ 
Update(BITree1, r+1, -val) \\ 
UpdateBIT2(BITree2, l, val*(l-1)) \\ 
UpdateBIT2(BITree2, r+1, -val*r) \\

Range Sum \\
getSum(BITTree1, k) *k) - getSum(BITTree2, k) \\ 

% \subsection{Pragmas}
% \begin{itemize}
% 	\item \lstinline{#pragma GCC optimize ("Ofast")} will make GCC auto-vectorize for loops and optimizes floating points better (assumes associativity and turns off denormals).
% 	\item \lstinline{#pragma GCC target ("avx,avx2")} can double performance of vectorized code, but causes crashes on old machines.
% 	\item \lstinline{#pragma GCC optimize ("trapv")} kills the program on integer overflows (but is really slow).
% \end{itemize}
% \kactlimport{BumpAllocator.h}
% \kactlimport{SmallPtr.h}
% \kactlimport{BumpAllocatorSTL.h}
% \kactlimport{Unrolling.h}
% \kactlimport{SIMD.h}